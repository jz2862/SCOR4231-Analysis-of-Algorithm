\documentclass{article} 
\usepackage[margin=1in]{geometry}
\usepackage{graphicx}
\usepackage{amsmath}
\usepackage{colortbl}
\usepackage{xcolor}
\usepackage{listings}
\usepackage{algorithm}
\usepackage{algorithmic}

\lstset{ %
    language=Python,                % the language of the code
    basicstyle=\footnotesize,           % the size of the fonts that are used for the code
    numbers=left,                   % where to put the line-numbers
    numberstyle=\tiny\color{gray},  % the style that is used for the line-numbers
    stepnumber=2,                   % the step between two line-numbers. If it's 1, each line 
                                    % will be numbered
    numbersep=5pt,                  % how far the line-numbers are from the code
    backgroundcolor=\color{white},      % choose the background color. You must add \usepackage{color}
    showspaces=false,               % show spaces adding particular underscores
    showstringspaces=false,         % underline spaces within strings
    showtabs=false,                 % show tabs within strings adding particular underscores
    frame=single,                   % adds a frame around the code
    rulecolor=\color{black},        % if not set, the frame-color may be changed on line-breaks within not-black text (e.g. commens (green here))
    tabsize=2,                      % sets default tabsize to 2 spaces
    captionpos=b,                   % sets the caption-position to bottom
    breaklines=true,                % sets automatic line breaking
    breakatwhitespace=false,        % sets if automatic breaks should only happen at whitespace
    title=\lstname,                 % show the filename of files included with \lstinputlisting;
                                    % also try caption instead of title
    keywordstyle=\color{blue},          % keyword style
    commentstyle=\color{dkgreen},       % comment style
    stringstyle=\color{mauve},         % string literal style
    escapeinside={\%*}{*)},            % if you want to add LaTeX within your code
    morekeywords={*,...}               % if you want to add more keywords to the set                                   
}

\Large
\title{CSOR W4231 Analysis of Algorithm} 
\author{Jialin Zhao} 
\date{\today}
\begin{document} 
\maketitle{} 
\section{Problem \uppercase\expandafter{\romannumeral1}} 
\begin{table}[!hbp]
\LARGE
\begin{center}
\begin{tabular}{*{7}{|c}|}

\hline $\mathit{f}$&$\mathit{g}$&$\mathit{O}$&$\mathit{o}$&$\Omega$&$\omega$&$\Theta$\\
\hline $\log^2n$&$6\log n$&No&No&Yes&Yes&No\\
\hline $\sqrt{\log n}$&$(\log \log n )^3$&No&No&Yes&Yes&No\\
\hline $4n\log n$&$n\log 4n$&Yes&No&Yes&No&Yes\\
\hline $n^{3/5}$&$\sqrt{n}\log n$&No&No&Yes&Yes&No\\
\hline $5\sqrt{n}+\log n$&$2\sqrt{n}$&Yes&No&Yes&No&Yes \\
\hline $n^54^n$&$5^n$&Yes&Yes&No&No&No\\
\hline $\sqrt{n}2^n$&$2^{n/2+\log n}$&No&No&Yes&Yes&No\\
\hline $n\log 2n$&$\frac{n^2}{\log n}$&Yes&Yes&No&No&No\\
\hline $n!$&$2^n$&No&No&Yes&Yes&No\\
\hline $\log n!$&$\log n^n$&Yes&No&Yes&No&Yes\\
\hline
\end{tabular}
\end{center}
\end{table}

\section{Problem \uppercase\expandafter{\romannumeral2}}
\subsection{(a)}
\textbf{\large Correctness:}\\\\
\textbf{Base case:} i = 0\\
After the loop, $z_0=a_0$\\
The solution of the polynomial is true.\\
\textbf{Induction hypothesis:} Assume that the statement is true for case $i\geq 0$.\\
\textbf{Inductive step:} Show it true for case i+1\\
This case equals to case i, when we set the initial $z = a_{i+1}x+a_{i}$ instead of $z = a_{i}$\\
By the induction hypothesis, case i is correct which means $z = a_0 + a_1x + ... + a_ix^i$\\
Then in the case i+1, $z = a_0 + a_1x + ... + (a_{i+1}x+a_i)x^i = a_0 + a_1 + ... + a_ix^i + a_{i+1}x^{i+1}$\\
Which means the solution of the polynomial is also true in case i+1.\\
\textbf{Conclusion:}\\
It follows that the statement is true for all n since we can apply the inductive step for n = 2; 3; : : :
\subsection{(b)}
This function uses n multiplications and n additions, where n equals the number of iterations of the for loop.\\
\textbf{A special polynomial:}\\
A single non-zero coefficient $a_ix^i$\\
\textbf{Better alternative method:}\\
Using the repeated squaring method of evaluating $x^i$:\\
Repeated squaring method:\\\\
\textbf{\large Pseudocode:}
\begin{algorithm}
  \caption{Function exp-by-squaring$(x,i)$}
  \label{alg1}
  \begin{algorithmic}
  \IF{$i < 0$}
  \RETURN exp-by-squaring(1/x, -i)
  \ELSIF{i = 0}
  \RETURN 1
  \ELSIF{i = 1}
  \RETURN x
  \ELSIF{i is even}
  \RETURN exp-by-squaring(x * x,  i / 2)
  \ELSIF{i is odd}
  \RETURN x * exp-by-squaring(x * x, (i - 1) / 2)
  \ENDIF
  \end{algorithmic}
\end{algorithm}
\begin{algorithm}
  \caption{Function A-exp-by-squaring$(a,x,i)$}
  \label{alg1}
  \begin{algorithmic}
  \RETURN  a * exp-by-squaring(x,i)
  \end{algorithmic}
\end{algorithm}\\
\textbf{\large Time Complexity:}\\\\
The problem is converted into a half-scale subproblem and using 1 or 2 multiplation in O(1) time:\\
$$T(n)=T(n/2) + O(1)$$
According to master thereom, the method Will use O($\log i$) multiplications.\\\\
\textbf{\large Space Complexity:}\\\\
The depth of recursion tree is $\log n$, the space used by each recuirsion step is $O(1)$.\\
So, the space complexity is $O(\log n)$\\\\
\textbf{\large Correctness:}\\\\
\textbf{Base case:} n = 0,1\\
$$x_0=1,x_1=x$$
The statement is true.\\
\textbf{Induction hypothesis:} Assume that the statement is true for every case n from 0 to i,$i \ge 1$.\\
\textbf{Inductive step:} Show it true for case i+1\\
Since we can find a case i/2 or (i+1)/2 in which the statement is true as a base case.\\
If i+1 is an odd number, $x^{i+1}=x*{(x^2)}^{(i)/2}=x^{i+1}$\\
If i+1 is an even number, $x^{i+1}={(x^2)}^{(i+1)/2}=x^{i+1}$\\
Which means the statement is also true in case i+1.\\
\textbf{Conclusion:}\\
It follows that the statement is true for all n since we can apply the inductive step for every odd or even number n = 2; 3; : : :



\section{Problem \uppercase\expandafter{\romannumeral3}}
Let $v$ also be represented by two halfs:\\
$$
v =
\begin{bmatrix}
    v_{h}\\
    v_{l}
\end{bmatrix}\\
$$
Where $v_h$ and $v_l$ are of n/2 scale.\\
Then:\\
$$
H_kv = 
\begin{bmatrix}
    H_{k-1} & H_{k-1}\\
    H_{k-1} & -H_{k-1} 
\end{bmatrix}\\
*
\begin{bmatrix}
    v_{h}\\
    v_{l}
\end{bmatrix}\\
=
\begin{bmatrix}
    H_{k-1}v_{h}+H_{k-1}v_{l}\\
    H_{k-1}v_{h}-H_{k-1}v_{l}
\end{bmatrix}\\
$$
\textbf{\large Correctness:}\\\\
\textbf{Base case:} n = $2^k$, k = 0\\
Then $H_0 v = v$\\
The statement is true.\\
\textbf{Induction hypothesis:} Assume that the statement is true for case $k\ge 0$.\\
\textbf{Inductive step:} Show it true for case k+1\\
Since the $H_k v$ is correct,\\
Then $H_{k+1} v = \begin{bmatrix}
    H_{k} & H_{k}\\
    H_{k} & -H_{k} 
\end{bmatrix}
*
\begin{bmatrix}
    v_{h}\\
    v_{l}
\end{bmatrix}
=
\begin{bmatrix}
    H_{k}v_{h}+H_{k}v_{l}\\
    H_{k}v_{h}-H_{k}v_{l}
\end{bmatrix}$
is also correct.\\
Which means the statement is also true in case k+1.\\
\textbf{Conclusion:}\\
It follows that the statement is true for all n since we can apply the inductive step for every odd or even number k = 1; 2; 3; : : :\\\\
\textbf{\large Time Complexity:}\\\\
So each step we convert the problem of $H_{k}v$$(T(n)$ time) into $H_{k-1}v_{h}$($T(n/2)$ time) and $H_{k-1}v_{l}$($T(n/2)$ time) and some addition or substraction($O(n)$ time), since elementwise operations take $O(1)$ time.\\
Therefore:
$$
T(n)=2T(n/2)+O(n)\\
$$
According to Master Teorem:
$$
T(n)=\Theta (n\log n)<\Theta (n^2)
$$
where $n=2^k$.\\
Faster than the straight forward algorithm.\\\\
\textbf{\large Space Complexity:}\\\\
The depth of the recursion tree is $k=(\log_2{n})$, each recursion uses $O(n)$ space, so the space complexity should be computed by adding the space used in the deepest recursion branch: $\sum_{1}^{\log_2 n} O(n/{2^i})=O(n)$\\\\
\textbf{\large Pseudocode:}
\begin{algorithm}
  \caption{Function Hadamard$(k,v)$}
  \label{alg1}
  \begin{algorithmic}
  \IF{k = 0}
  \RETURN v
  \ENDIF
  \STATE $v_h$ $\Longleftarrow$ Higher Half Part of v
  \STATE $v_l$ $\Longleftarrow$ Lower Half Part of v
  \STATE A $\Longleftarrow$ Hadamard$(k-1,v_h)$
  \STATE B $\Longleftarrow$ Hadamard$(k-1,v_l)$
  \STATE Result $\Longleftarrow$ $\begin{bmatrix}
  A+B\\
  A-B
  \end{bmatrix}$
  \RETURN Result
  \end{algorithmic}
\end{algorithm}




\section{Problem \uppercase\expandafter{\romannumeral4}}
\subsection{(a)}
If we use divide and conquer method to get a $O(n\log n)$-time solution.\\
We can divide the problem into two (n/2) scale subproblem and use $O(n)$ time to combine subproblems.\\
Here is my sample solution:\\\\
\textbf{\large Pseudocode:}
\begin{algorithm}
  \caption{Function Majority(array $A$)}
  \label{alg1}
  \begin{algorithmic}
  \IF{$Length(A)=0$}
  \RETURN None
  \ENDIF
  \IF{$Length(A)=1$}
  \RETURN the only element in A
  \ENDIF
  \STATE $A1$ $\Longleftarrow$ Array of first n/2 elements of $A$
  \STATE $A2$ $\Longleftarrow$ Array of the left elements of $A$
  \STATE $a1$ $\Longleftarrow$ Majority($A1$)
  \STATE $a2$ $\Longleftarrow$ Majority($A2$)
  \IF{($a1$ is None) and ($a2$  is None)}
  \RETURN None
  \ELSIF{($a1$ is not None) and ($a2$ is not None) and ($a1$ = $a2$)}
  \RETURN a1
  \ELSIF{$a1$ is not None}
  \STATE num $\Longleftarrow$ the amount of $a1$ in A
  \IF{num$>\lfloor 1/2Length(A)\rfloor$}
  \STATE return $a1$
  \ENDIF
  \ELSIF{$a2$ is not None}
  \STATE num $\Longleftarrow$ the amount of $a2$ in A
  \IF{num$>\lfloor 1/2Length(A)\rfloor$}
  \STATE return $a2$
  \ENDIF
  \ELSE\RETURN None
  \ENDIF
  \end{algorithmic}
\end{algorithm}\\
\textbf{\large Correctness:}\\\\
As stated, the method is based on divide-conquer principle by induction on the size of the two lists.\\
If A has a majority element v, v must also be a majority element of A1 or A2 or both.\\
If both parts have the same majority element, it is automatically the majority element for A.\\ If one or two of the parts has a different majority element, count the number of that element in both parts (in O(n) time) to check if it is a majority element.\\
If not the cases above, return None.\\
\textbf{Base case:} subarray of length k = 1\\
Will return the only element which is definitely the majority.\\
The statement is true.\\
\textbf{Induction hypothesis:} Assume that the statement is true for case of length $k\ge length \ge 1$.\\
\textbf{Inductive step:} Show it true for case of combined array of length k+1\\
According to the base case, the two subarray of this case is in length (k+1)/2 and will return their majority number correctly.\\
After choosing and checking the two potential majority numbers, the array in this case can also return its majority number correctly.\\
\textbf{Conclusion:}\\
It follows that the statement is true for all array since we can apply the inductive step for any length k = 1; 2; 3; : : :\\\\
\textbf{\large Time Complexity:}\\\\
A recurrence relation is T(n) = 2T(n/2) + O(n)\\
According to Master Teorem:
$$
T(n)=O (n\log n)\\
$$
\textbf{\large Space Complexity:}\\\\
The depth of the recursion tree is $\log n$, each recursion step uses $O(n/2^i)$ space.\\
So the space complexity should be calculated by $\sum_1^{\log n} O(n/{2^i})=O(n)$



\subsection{(b)}
If we can use a map storing pairs of (element,count), the problem will be simplified a lot because we can easily look through all elements in the array A, add new element into map and set the count as 0, or we let count+=1. Then just find the only element with a count more than half of the length of A.\\
However, let's suppose we need to use limited techniques(only array):\\
My solution is based on an intersting fact of majority number in an array: By discarding two different elements of A, the majority number will not change in the array left.\\
After discarding all different element pairs the remaining element should be the majority, unless A doesn't have a majority element at all.\\
So we need to check if the left element is the majority, if yes return it, if no return None.\\\\
\textbf{\large Correctness:}\\\\
\textbf{Base case:} Array of length k = 0,1\\
Will return None or the only element which is definitely the majority.\\
The result is true.\\
\textbf{Induction hypothesis:} Assume that the statement is true for case of length $k\ge length \ge 0$.\\
\textbf{Inductive step:} Show it true for case of array of length = k+2\\
After discarding two different elements in the array, the majority element number is reduced by 0 or 1 but the total length is reduced by 2, so the majority number will still exceed half number of the arrray.\\
So this case will return its majority if it has.\\
The result is true.\\
\textbf{Conclusion:}\\
It follows that the statement is true for all array since we can apply the inductive step for any length k = 1; 2; 3; : : :\\
\textbf{\large Time Complexity:}\\\\
A recurrence relation is T(n) = T(n-1) + O(1)\\
Therefore:
$$
T(n)=O (n)
$$
\textbf{\large Space Complexity:}\\\\
This is an recurrence method which needs only O(1) extra space.\\\\
\textbf{\large Pseudocode:}
\begin{algorithm}
  \caption{Function Majority(array $A$)}
  \label{alg1}
  \begin{algorithmic}
  \STATE N $\Longleftarrow$ Length of A
  \IF{N = 0}
  \RETURN None
  \ENDIF 
  \STATE Element $\Longleftarrow$ A[1]
  \STATE Count $\Longleftarrow$ 1
  \STATE i $\Longleftarrow$ 2
  \FOR{$i\le N$}
  \STATE i = i + 1
  \IF{Element = A[i]}
  \STATE Count $\Longleftarrow$ Count + 1
  \ELSIF{Count $\ge$ 1}
  \STATE Count $\Longleftarrow$ Count - 1
  \ELSIF{Count = 0}
  \STATE Element $\Longleftarrow$ A[i]
  \STATE Count $\Longleftarrow$ 1
  \ENDIF
  \ENDFOR
  \IF{Look through A and check: Element occurs more than half of the length of A}
  \RETURN Element
  \ELSE
  \RETURN None
  \ENDIF
  \end{algorithmic}
\end{algorithm}\\







\section{Problem \uppercase\expandafter{\romannumeral5}}
\subsection{(a)}
\begin{table}[!hbp]
\LARGE
\begin{center}
\begin{tabular}{*{9}{|c}|}
\hline $n$&$0$&$1$&$2$&$3$&$4$&$5$&$6$&$7$\\
\hline $F_n$&$0$&$1$&$1$&$2$&$3$&$5$&$8$&$13$\\
\hline
\end{tabular}
\end{center}
\end{table}
\textbf{\large Correctness:}\\\\
\textbf{Base case:} n = 6,7\\
$F_6=8 \ge 2_{n/2}=8$\\
$F_7=13 \ge 2_{n/2}=8\sqrt{2}$\\
The statement is true.\\
\textbf{Induction hypothesis:} Assume that the statement is true for case $n\geq 6$ and case $n+1$.\\
\textbf{Inductive step:} Show it true for case n+2\\
$F_{n+2}=F_{n+1}+F_{n} \ge 2^{(n+1)/2}+2^{(n)/2}\ge2*2^{(n)/2}=2^{(n+2)/2}$\\
Which means the statement is also true in case n+2.\\
\textbf{Conclusion:}\\
It follows that the statement is true for all $n\ge 6$ since we can apply the inductive step for n = 8; 9; : : :







\subsection{(b)}
\subsubsection{}
\textbf{\large Pseudocode:}
\begin{algorithm}
  \caption{Function Fibonacci$(n)$}
  \label{alg1}
  \begin{algorithmic}
  \IF{$n\le 1$}
  \RETURN n
  \ENDIF
  \RETURN Fibonacci(n-1)+Fibonacci(n-2)
  \end{algorithmic}
\end{algorithm}\\
\textbf{\large Correctness:}\\\\
This is the definition, no need to prove it.\\
Base case: n = 0,1 then $F_0=0,F_1=1$ is true.\\
Induction step: if case $n \ge 0$ and $n+1$ is true then case n+2 : $F_{n+2}$=$F_{n+1}$+$F_{n}$=true.\\
So the case n+2 is also true.\\ 
Conclusion: The Fabonacci number is calculated correctly for every case n =1;2;3:::\\\\
\textbf{\large Time Complexity:}\\\\
The problem of computing $F_n$ is converted to computing $F_{n-1}$ and $F_{n-2}$ and a constant-time addition.\\
Then we have: \\
$$T(n) = T(n-1)+T(n-2)+O(1)$$
So in this prograss, we only need to calculate the sum of adding two numbers:\\
From T(n) to T(n-1): 1 addition;\\
From T(n-1) to T(n-2): 2 addition;\\
From T(n-2) to T(n-3): 3 addition;\\
From T(n-3) to T(n-4): 5 addition;\\
.....\\
From T(n-x) to T(n-x-1): $F_{x+2}$ addition;\\
.....\\
From T(2) to T(1) and T(0): $F_{n}$ addition;\\
Then, $T(n)=\Theta (\sum_2^n F_i$)\\
According to the expression in question(a):
$$F_i \ge 2^{(n)/2}$$
Therefore: 
$$T(n)=\Theta(\sum_2^n F_i)=\Omega(\sum_2^n 2^{(n)/2})=\Omega (2^{(n)/2})$$
To give a tight bound:\\
We know from the book that $F_n$ has a closed form expression: $$\frac{\phi^n-\varphi^n}{\sqrt{5}}$$
where $\phi=\frac{1+\sqrt{5}}{2}$ and $\varphi=\frac{1-\sqrt{5}}{2}$\\
Therefore:
$$T(n)=\Theta(\sum_2^n F_i)=\Theta(\sum_2^n \phi^i)=\Theta(\phi^n)$$
\textbf{\large Space Complexity:}\\\\
The depth of the recursion tree is n, each recursion step uses $O(1)$ extra space.\\
The space complexity is $O(n)$.




\subsubsection{}
\textbf{\large Pseudocode:}
\begin{algorithm}
  \caption{Function Fibonacci$(n)$}
  \label{alg1}
  \begin{algorithmic}
  \STATE Memory $\Longleftarrow$ 0
  \STATE Result $\Longleftarrow$ 1
  \IF{$n\le 1$}
  \RETURN n
  \ENDIF 
  \FOR{$i = 2$ up to n}
  \STATE Temp $\Longleftarrow$ Result
  \STATE Result $\Longleftarrow$ Result + Memory
  \STATE Memory $\Longleftarrow$ Temp
  \ENDFOR
  \RETURN Result
  \end{algorithmic}
\end{algorithm}\\
\textbf{\large Correctness:}\\\\
This is the definition, no need to prove it.\\
Base case: n = 0,1 then $F_0=0,F_1=1$ is true.\\
Induction step: if case $n \ge 0$ and $n+1$ is true then case n+2 : $F_{n+2}$=$F_{n+1}$+$F_{n}$=true.\\
So the case n+2 is also true.\\ 
Conclusion: The Fabonacci number is calculated correctly for every case n =1;2;3:::\\\\
\textbf{\large Time Complexity:}\\
$$T(n)=T(n-1)+O(1)$$
To compute $F_n$, the algorithm needs n-1 loops, each loop needs 1 addition in $O(n)$. So we need $n-1*O(1)= O(n)$ time.\\\\
\textbf{\large Space Complexity:}\\\\
This is a recurrence method using $O(1)$ extra space.
\subsubsection{}
To prove
$$
\begin{bmatrix}
    {0} & {1}\\
    {1} & {1} 
\end{bmatrix}^N
=
\begin{bmatrix}
    {F_{N-1}} & {F_N}\\
    {F_N} & {F_{N+1}} 
\end{bmatrix}
$$
where $N\ge 1$\\
\textbf{Proof:}\\
Base case: N = 1\\
$$
\begin{bmatrix}
    {0} & {1}\\
    {1} & {1} 
\end{bmatrix}^1
=
\begin{bmatrix}
    {F_{0}} & {F_1}\\
    {F_1} & {F_{2}} 
\end{bmatrix}
$$
The statement is true in base case.\\
Induction hypothesis: Assume that the statement is true for case $N \ge 1$.\\
Inductive step:
$$
\begin{bmatrix}
    {0} & {1}\\
    {1} & {1} 
\end{bmatrix}^{N+1}
=
\begin{bmatrix}
    {0} & {1}\\
    {1} & {1} 
\end{bmatrix}^{N}
*
\begin{bmatrix}
    {0} & {1}\\
    {1} & {1} 
\end{bmatrix}
=
\begin{bmatrix}
    {F_{N-1}} & {F_N}\\
    {F_N} & {F_{N+1}} 
\end{bmatrix}
*
\begin{bmatrix}
    {0} & {1}\\
    {1} & {1} 
\end{bmatrix}
=
\begin{bmatrix}
    {F_{N}} & {F_{N+1}}\\
    {F_{N+1}} & {F_{N+2}} 
\end{bmatrix}
$$
The statement is also true in case N+1.
Conclusion:
It follows that the statement is true for all n since we can apply the inductive step for n = 2; 3; : : :\\
-----------------------------\\
The $O(\log n)$ time method is:	\\\\
\textbf{\large Pseudocode:}
\begin{algorithm}
  \caption{Function FabonacciMatrix$(A,n)$}
  \label{alg1}
  \begin{algorithmic}
  \IF{n = 1}
  \RETURN A
  \ELSIF{n = 2}
  \RETURN A*A
  \ELSIF{n is even}
  \RETURN FabonacciMatrix$(A*A,n/2)$
  \ELSIF{n is odd}
  \RETURN A*FabonacciMatrix$(A*A,(n-1)/2)$
  \ENDIF
  \end{algorithmic}
\end{algorithm}
\begin{algorithm}
  \caption{Function Fabonacci$(n)$}
  \label{alg1}
  \begin{algorithmic}
  \STATE A $\Longleftarrow \begin{bmatrix}
    {0} & {1}\\
    {1} & {1} 
\end{bmatrix}$ 
  \IF{n $\le$ 1}
  \RETURN n
  \ELSE
  \STATE F $\Longleftarrow$ FabonacciMatrix$(A,n)$
  \RETURN F[0][1]
  \ENDIF
  \end{algorithmic}
\end{algorithm}\\
\textbf{\large Time Complexity:}\\\\
So each step we convert the problem of $A^n$$(T(n)$time) into $A^{n/2}$$(T(n/2)$time) and one or two 2*2 matrix multiplition($O(1)$time)), since elementwise operations take $O(1)$ time.\\
Therefore:
$$
T(n)=2T(n/2)+O(1)\\
$$
According to Master Teorem:
$$
T(n)=O (\log n)
$$
\textbf{\large Space Complexity:}\\\\
The depth of the recursion tree is $\log n$, each recursion step takes $O(1)$ extra space.\\
The space complexity is $O(\log n)$

\subsection{(c)}
Because we are multiplying numbers that have value on the order of $\phi^n$, hence have order n bits.\\
(b)1.\\Beacause we only use integer addition in this reccursion method, so:
$$T(n)=T(n-1)+T(n-2)+\Theta(n)$$
Since we know the number of addition is not changed, therefore:
$$T(n)=\Theta(n\sum_2^n F_i)=\Theta(n\sum_2^n \phi^i)=\Theta(n\phi^n)=O(2^n)$$
(b)2.\\Beacause we only use integer addition in this iteration method, so:
$$T(n)=T(n-1)+\Theta(n)$$
Therefore, $$T(n)=(n-1)*\Theta(n)=\Theta(n^2)$$
(b)3.\\We are using only multiplation of time $\Theta(n^2)$ in every loop, so:
$$T(n)=T(n/2)+\Theta(n^2)$$
According to master theorem:
$$T(n)=\Theta(n^2)$$






\end{document}