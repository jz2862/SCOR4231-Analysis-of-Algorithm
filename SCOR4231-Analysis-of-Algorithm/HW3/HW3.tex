\documentclass{article} 
\usepackage[margin=1in]{geometry}
\usepackage{graphicx}
\usepackage{amsmath}
\usepackage{colortbl}
\usepackage{xcolor}
\usepackage{listings}
\usepackage{algorithm}
\usepackage{algorithmic}
\usepackage{tikz}
\usepackage{graphicx}
\usepackage{float}

\Large
\title{CSOR4231 Algorithm\\HW3} 
\author{Jialin Zhao --- jz2862} 
\date{\today}
\begin{document} 
\maketitle{} 
\section{Problem \uppercase\expandafter{\romannumeral1}} 
\textbf{Find the longest monotonically increasing subsequence of a sequence.\\ Give a dynamic programming algorithm that solves this problem in time $O(n^2)$.\\\\}
Subproblem OPT(i) = L[i] is the length of the LIS subsequence ending at $a_i$.\\
The recurrence relationship is $OPT(i) = \max\limits_{1\le k<i,a_k\le a_i}(OPT(k)+1,1)$.\\
The problem equals to finding max(OPT(i)), i.e. the max length of LIS subsequnece ending at any i-th element,$i \in [1,n]$.\\\\
\textbf{\large Pseudocode:\\}
Represent the input sequence as A = $[a_1,a_2...a_n]$.\\
Here I use an array L[n] where L[i] = OPT(i) to denote the length of the LIS subsequence ends at $a_i$.\\
And I use an array F[n] to denote the position number of the former element in the LIS subsequence ends at $a_i$, aiming to backtrack the subsequence.\\
MaxValue is the length of LIS; Result is the LIS.\\
\begin{algorithm}[H]
  \caption{Function LIS$(A)$}
  \label{alg1}
  \begin{algorithmic}
  \STATE L $\Longleftarrow$ an array of the length n while filled with element 1
  \STATE F $\Longleftarrow$ an array of the length n while filled with element 0
  \STATE MaxValue $\Longleftarrow$ 0
  \STATE Pos $\Longleftarrow$ 0
  \FOR{i from 2 to n}
  \FOR{j from 1 to i-1}
  \IF{$A[j-1] \le A[i-1]$}
  \IF{L[i-1] $<$ L[j-1]+1}
  \STATE L[i-1] = L[j-1]+1
  \STATE F[i-1] = j-1
  \ENDIF
  \ENDIF
  \ENDFOR
  \IF{MaxValue $<$ L[i-1]}
  \STATE MaxValue = L[i-1]
  \STATE Pos = i-1
  \ENDIF
  \ENDFOR
  \STATE Result $\Longleftarrow$ an array of the length MaxValue and filled with element Pos
  \FOR{i from 2 to Maxvalue}
  \STATE Result[MaxValue-i]=F[Pos]
  \STATE Pos = F[Pos]
  \ENDFOR
  \STATE Result $\Longleftarrow$ convert the element position stored in Result to the corresponding value in A
  \RETURN MaxValue, Result
  \end{algorithmic}
\end{algorithm}
\noindent\textbf{\large Correctness:\\}
\textbf{Base case: } OPT(0) = 0;\\
\textbf{Induction:} \\
So if we suppose the case $OPT(i-1),i \ge 1$ is correct for the length of longest monotonically increasing subsequence ebds at $a_{i-1}$;\\
$OPT(i) = \max\limits_{1\le k<i,a_k\le a_i}(OPT(k)+1,1)$.\\
As we know, for the LIS subsequence ends at $a_i$, $a_i$ must following a LIS subsequence ends at $a_j$ where $j < i$ and $a_j \le a_i$; if there is no former element smaller than $a_i$, the LIS subsequence ends at $a_i$ is just $a_i$ itself with length 1.\\
Since max($(OPT(k)+1)\ (for\ 1\le k<i,a_k\le a_i)\ ,\ 1$) covers every case that may satisfy the above statement.\\
Then OPT(i) is also the length of the longest monotonically increasing subsequence.\\
\textbf{Conclusion:\\}
It follows that the statement is true for all $1\le i \le n$ since we can apply the inductive step for every odd or even number i = 2; 3; : : :\\\\
\textbf{\large Runnig time:\\}
Outside the DP loop: Running time is O(n)+O(M) = O(n) where M is the length of the LIS sequence, $M\le n$.\\
Inside the DP loop: The total number of the iterations is  $\sum_{i=1}^n i = O(n^2)$;The time for every iteration is O(1). So the running time costed inside the loop is $O(n^2).$\\\\
The running time is $O(n^2).$\\\\
\textbf{\large Space complexity:\\}
The additional space in this algorithm is the space of L,F,Result,MaxValue,Pos.\\
The space complexity is $O(n)$


\clearpage
\section{Problem \uppercase\expandafter{\romannumeral2}} 
\textbf{Give an efficient algorithm that determines:\\
At which libraries to place copies of the books so that the total cost is minimized; The minimum total cost.}\\
OPT(i) is the minimum cost for the subproblem that considers libraries from 1 to i.\\
The recurrence relationship is $OPT(j) = \min\limits_{1\le k < i}(OPT(k)+Cost(k+1,j)$),the\ Cost(k,j) = 0.5*(j-k)*(j-k+1) is the total cost from k to j\ when\ only\ j\ store\ a\ book.\\
The problem equals to finding $OPT(n)$.\\\\
\textbf{\large Pseudocode:}\\
Denote the cost of storing the book in each library as array L = $[c_1,c_2...c_n]$, use the similiar DP strategy as matrix chain multiplication.\\
Use array C[i] of size n, C[i] = OPT(i).\\
Use array F[i] of size n to denote the former library storing the book before library i in the best strategy.\\
The algorithm iterate on the DP recurrence relationship mentioned above and get the minimum cost and distribution.
\begin{algorithm}[H]
  \caption{Function MininumCost$(L)$}
  \label{alg1}
  \begin{algorithmic}
  \STATE C $\Longleftarrow$ an array of size n while filled with $\infty$
  \STATE F $\Longleftarrow$ an array of size n while filled with -1
  \FOR{i from 1 to n}
  \FOR{k from 1 to i}
  \STATE q = C[k-1] + 0.5 * (i-k)(i-k-1) + L[i-1]
  \IF{$q<C[i-1]$}
  \STATE C[i-1]=q
  \STATE F[i-1]=k
  \ENDIF
  \ENDFOR
  \ENDFOR
  \RETURN $C[0][n-1]$, the order by backtrack through F starting at F[n-1] until finding -1
  \end{algorithmic}
\end{algorithm}
\noindent\textbf{\large Correctness:\\}
\textbf{Base case: } for i = 1, $OPT(1) = c_1$ is the correct cost\\
\textbf{Induction}: \\
Suppose the case $OPT(x)\ for\ every\ x \in [1,i], i\ge 1$ is true that it represents the minimum cost of the book storage distribution from 1 to i.\\
For $OPT(i+1):$\\
Since we know that the user delay of a library is merely decided by the nearest following library storing a book, so the cost for library i to j can be divided into two seperate part if there is a last but not one library k storing a book: Cost(i+1) = Cost(i,k)+Cost(k+1,j). So the $OPT(i) = \min\limits_{1\le k \le i}(OPT(k)+Cost(k+1,j))$ considers every possible distribution, and is also true for being the minimum cost of the book storage distribution from i to j.\\
\textbf{Conclusion:}:\\
It follows that the statement is true for all $i \in [1,n]$ since we can apply the inductive step for every odd or even number i = 2; 3; : : :\\\\
\textbf{\large Runnig time:\\}
Outside the DP loop: Since the running time of backtrak(F) is O(n), the total time cost is O(n)\\
Inside the DP loop: The total number of the iterations is  $O(n^2)$;\\
The time for every iteration is O(1). So the running time costed inside the loop is $O(n^2).$\\\\
The running time is $O(n^2).$\\\\
\textbf{\large Space complexity:\\}
The additional space in this algorithm is the $O(n)$ space for C,F.\\
The space complexity is $O(n)$



\clearpage
\section{Problem \uppercase\expandafter{\romannumeral3}}
\textbf{Give an efficient algorithm that determines an ordering of the tasks for the supercomputer
that minimizes the duration of the process, as well as the minimum duration.}\\
A greedy algorithm is needed for this problem.\\
To minimize the end time of the processors, we need to schedule the jobs in decreasing order of $f_i$.\\\\
\textbf{\large Pseudocode:}\\
Here I use P=[$p_1,p_2,p_3...p_n$] to denote the supercomputer time for each task and F=[$f_1,f_2,f_3...f_n$] to denote the processor time for each task.\\
\begin{algorithm}[H]
  \caption{Function Greedy$(P,F)$}
  \label{alg1}
  \begin{algorithmic}
  \STATE //Keep track of the index while we do the sort on the descending order of $f_i$:
  \FOR{i from 1 to n}
  \STATE F[i-1]=(F[i-1],i) 
  \ENDFOR
  \STATE MergeSort(F) on the descending order of $f_i$      //i.e. F[i-1][0]
  \STATE Order $\Longleftarrow$ the order of index in F      //i.e. F[i-1][1]
  \STATE Duration = 0, Start = 0
  \FOR{i from 1 to n}
  \STATE Start += P[F[i-1][1]-1]
  \STATE Duration = max(Duration,Start+F[i-1][0])
  \ENDFOR
  \RETURN Order, Duration
  \end{algorithmic}
\end{algorithm}
\noindent\textbf{\large Correctness:\\}
Note that no matter what the order is, the duration ends with the last task finished by a processor.\\
The endtime is the maximum of the endtime of each processor:\\
= whole time spent by supercomputer + max(the remaining time of the processor)\\
We know that the whole time spent by supercomputer is always tha same $\sum_1^n p_i$. So we need to put the most processor-time-consuming task at the first place followed by the second most processor-time-consuming task......\\
To put it another way, longer-on-processor-time tasks  need to be arranged before shorter-on-processor-time task.\\
By contradiction, if we swap the order of two tasks in our result, the finishing time for those two task will be determined by later one which means the processing time will be longer or un-changed.\\
\textbf{Conclusion:}\\
So the order in the result is optimal.\\
\textbf{\large Runnig time:\\}
The MergeSort takes $O(n\log n)$ and other parts take $O(n)$ time.\\\\
The running time is $O(n\log n).$\\\\
\textbf{\large Space complexity:\\}
The additional space in this algorithm is:\\
The MergeSort uses space $O(n)$;\\
Order uses $O(n)$ space; other elements use $O(1)$ space.\\\\
The space complexity is $O(n)$\\





\clearpage
\section{Problem \uppercase\expandafter{\romannumeral4}}
\textbf{Design and analyze a dynamic programming algorithm for this problem that runs in time polynomial in n and $\sum_{i=1}^n a_i$}\\
Subproblem boolean OPT(i,j,k) represents if there are two disjoint subsets I,J $\subseteq [a_1...a_k]$ satisfying the sum of I is x and the sum of J is y.\\
The recurrence relationship is $OPT(i,j,k) = OPT(i,j,k-1)\cup OPT(i-a_i,j,k)\cup OPT(i,j-a_i,k)$\\
Set $A =1/3\sum_{i=1}^n a_i$, if A is an integer and we can find two disjoint subset of L whose sum = A, then the left elements should also sum to A.\\
The 3-partition problem equals to find true or false in OPT(n,A,A)\\
Define a boolean matrix OPT(i,j,k)=M[i][j][k] in size [A+1][A+1][n+1], with the meaning that M[x, y, k] where $x,y \in [0,A],k \in [0,n]$.\\
As we know that if the M[i][j][k]=1, M[i][j][k+1] should also be 1, the true value can be inherited from former matrix, so we only need to maintain matrix M[A+1][A+1] and update it for n+1 iterations.\\\\
\textbf{\large Pseudocode:}
\begin{algorithm}[H]
  \caption{Function 3-Partition$(L)$}
  \label{alg1}
  \begin{algorithmic}
  \STATE n = length of L
  \STATE $A =1/3\sum_{i=1}^n L[i-1]$ 
  \IF{A is not an integer or n is less than 3}
  \RETURN false
  \ENDIF
  \STATE M $\Longleftarrow$ a matrix of size (A+1,A+1) and filled with false
  \STATE M[0][0]=true
  \FOR{k from 1 to n}
  \STATE a = L[k-1]
  \FOR{i from 0 to A}
  \FOR{j from 0 to A}
  \STATE M[i][j] = M[i][j] or M[i-a][j] or M[i][j-a]
  \ENDFOR
  \ENDFOR
  \ENDFOR
  \RETURN M[A][A]
  \end{algorithmic}
\end{algorithm}
\noindent\textbf{\large Correctness:\\}
\textbf{Base case:} $$OPT(i,j,k) =\left\{
\begin{aligned} 
 & 1(when\ i = 0,j=0)\\
 & 0(else)
\end{aligned}\right.
$$
is the correct representation for k = 0 whether there are two disjoint subsets I,J $\subseteq [a_1...a_k]$ satisfying the sum of I is x and the sum of J is y.\\
\textbf{Induction:} Suppose OPT(i,j,k-1) is correct for $k \ge 1$ \\
$OPT(i,j,k) = OPT(i,j,k-1)\cup OPT(i-a_k,j,k)\cup OPT(i,j-a_k,k)$\\
We know that if we can find two subsets meeting the requirment taking the 1 to k-th elements into consideratoin, when we remove the k th element, we can also find two valid subsets for 1 to k-1th elements;otherwise there should be two valid subsets sum to $i-a_k$,$j$ or $i,j-a_k$ for 1 to k-1 th elements. And those information is correctly known in our asumption.\\
Then the representation is also correct for OPT(i,j,k).\\
\textbf{Conclusion:} \\
The OPT(i,j) in the algorithm is true for every $i \in [1,n], j \in [1,D]$\\\\
\textbf{\large Runnig time:\\}
Outside the DP loop: O(n)\\
Inside the DP loop: The total number of the iterations is  $\sum_{k=1}^n A^2  = O(nA^2)$;\\
The time for every iteration is O(1). So the running time costed inside the loop is $O(nA^2).$\\\\
The running time is $O(nA^2).$\\\\
\textbf{\large Space complexity:\\}
The additional space in this algorithm is the $O(A^2)$ space for M.\\\\
The space complexity is $O(A^2)$


\clearpage
\section{Problem \uppercase\expandafter{\romannumeral5}}
\textbf{You are given some data to analyze. You can spend D dollars to perform the
analysis. You have organized the process of analyzing the data so that it consists of n tasks
that have to be performed sequentially by using dedicated hardware: you will use a processor
$P_i$ to perform task i, for every i. Each processor is relatively cheap but may fail to complete
its task with some probability, independently of the other processors. Specifically, $P_i$ costs $c_i$
dollars and succeeds to complete its task with probability $s_i$, while it fails with probability
$1 - s_i$.}
\subsection{What is the probability that the process of analyzing the data will be com-
pleted successfully?}
According to chain rules:  the probability that the process of analyzing the data will be com-
pleted successfully $P_a=s_1*s_2*...s_n=\prod_{i=1}^n s_i$\\
\subsection{Note that you can improve this success probability by using $p_i$ identical processors $P_i$
for task i instead of just one.}
\subsubsection{What is the probability that task i will be completed successfully now?}
$P_i = 1- probability\ to\ fail = 1- (1-s_i)^{p_i}$
\subsubsection{What is the probability that the process of analyzing the data will be completed successfully?}
According to chain rules: the probability that the process of analyzing the data will be completed successfully $P = \prod_{i=1}^n P_i = \prod_{i=1}^n (1-(1-s_i)^{p_i})$

\subsubsection{Given $s_1$; : : : ; $s_n$; $c_1$; : : : ; $c_n$ and D, compute $p_1$; : : : ; $p_n$ such that the success probability of the entire process is maximized while you do not spend more than D dollars.}Subproblem $OPT(i,j)$ denotes the maximum success probability for task 1 to i if we spend at most j money on them.\\
The recurrence relationship is: 
$$ OPT(i,j)=\max \left\{
\begin{aligned}
& OPT(i,j-c_i)*(1-(1-s_i)^{1}) \\
& \dots \dots\\
& OPT(i,j-(k-1)c_i)*(1-(1-s_i)^{k-1}) \\
& OPT(i,j-kc_i)*(1-(1-s_i)^{k})
\end{aligned}
\right\}
$$
where k satisfy that $j-kc_i$ can still allocate enough money for at least 1 processor for each task before i.\\
So the problem is equal to finding the OPT(1,n)\\\\
\textbf{\large Pseudocode:\\}
% \begin{algorithm}[H]
%   \caption{Function distribution$(S,C,D)$}
%   \label{alg1}
%   \begin{algorithmic}
%   \STATE n $\Longleftarrow$ length of S
%   \STATE Basic $\Longleftarrow$ 0
%   \STATE/* we know that every task need at least 1 processor as base */
%   \STATE/* Basic is the least cost for task 1 to n-1 */
%   \FOR{i from 1 to n-1}
%   \STATE Basic+= C[i-1]
%   \ENDFOR
%   \STATE N = (D-Basic)/C[n-1]
%   \STATE dist $\Longleftarrow$ array of size n filled with 0
%   \STATE prob $\Longleftarrow$ 0
%   \FOR{i from 0 to N }
%   \STATE iProb = $1-(1-S[n-1])^i$
%   \STATE Remain = D-i*C[n-1]
%   \STATE dist0, prob0 = distribution(S[0:n-2],C[0:n-2],Remain)
%   \IF{prob0*iProb$>$prob}
%   \STATE prob = prob0*iProb
%   \STATE dist = dist0.append(i)
%   \ENDIF
%   \ENDFOR
%   \RETURN dist,prob
%   \end{algorithmic}
% \end{algorithm}
I use K[i][j] of size (n+1,D+1) as OPT(i,j).
Basice as the value 

\begin{algorithm}[H]
  \caption{Function distribution$(S,C,D)$}
  \label{alg1}
  \begin{algorithmic}
  \STATE K $\Longleftarrow$ Matrix of size (n+1,D+1) to record the max probablity from task 1 to task i and is filled with 0 
  \STATE Dist $\Longleftarrow$ Matrix of size (n+1,D+1), the best number of processor i for max probablity from task 1 to task i, filled with 0
  \STATE n $\Longleftarrow$ length of S
  \STATE/* Basic is the least cost for task 1 to i-1,we know that every task need at least 1 processor */
  \STATE Basic $\Longleftarrow$ array of length n+1
  \STATE Basic[0] = 0, Basic[1] = 0
  \FOR{i from 2 to n}
  \STATE Basic[i] = Basic[i-1]+C[i-2]
  \ENDFOR
  \FOR{i from 0 to n}
  \FOR{j from 0 to D}
  \IF{i==0 or j==0}
  \STATE K[i][j]=0
  \STATE continue
  \ENDIF
  % \IF{j$<$Basic[i]+C[i-1]}
  % \STATE continue
  % \ENDIF
  \FOR{x from 1 to (j-Basic[i])/C[i-1]}
  \STATE xProb = $1-(1-S[i-1])^x$
  \STATE xCost = x*C[x-1]
  \STATE prob = K[i-1][j-xCost]*xProb
  \IF{prob$>$K[i][j]}
  \STATE K[i][j]=prob
  \STATE Dist[i][j] = x
  \ENDIF
  \ENDFOR
  \ENDFOR
  \ENDFOR
  \RETURN K[n][D], backtrack(C,Dist,n,D)
  \end{algorithmic}
\end{algorithm}
\begin{algorithm}[H]
  \caption{Function backtrack(C,Dist,n,D)}
  \label{alg1}
  \begin{algorithmic}
  \STATE Distribution $\Longleftarrow$ an array of length n
  \FOR{i from n to 1}
  \STATE Result[i-1] = Dist[i][D]
  \STATE D = D - Dist[i][D]*C[i-1]
  \ENDFOR
  \end{algorithmic}
\end{algorithm}
\noindent\textbf{\large Correctness:\\}
\textbf{Base case:} OPT(i,j)=0 when i=0 or j=0; It's true for being the maximum probability for the first i tasks under j money.\\
\textbf{Induction:} Suppose the maximum probability is correct for every $OPT(x\le i,y\le j)$ except OPT(i,j).\\
For OPT(i,j), we know that the any valid distribution for i,j must allocate an amount of money on i and remain some money on tasks before i.\\
So the maximum probablity for task 1 to i must be multiplation of maximum probability for task 1 to i-1 and probability for task i, under some allocation of money.\\
Since every best distribution for tasks before i with money less than j, i.e. $OPT(x\le i,y\le j)$, is correctly known.\\
Then the maximum probablity is also correct for OPT(i,j).\\
\textbf{Conclusion:} \\
The OPT(i,j) in the algorithm is true for every $i \in [1,n], j \in [1,D]$\\\\
\textbf{\large Runnig time:\\}
Outside the DP loop: O(n)\\
Inside the DP loop: The total number of the iterations is  $\sum_{i=1}^n \sum_{j=0}^{D} ((j-Basic[i])/C[i-1])  = O(nD^2)$;\\
The time for every iteration is O(1). So the running time costed inside the loop is $O(nD^2).$\\\\
The running time is $O(nD^2).$\\\\
\textbf{\large Space complexity:\\}
The additional space in this algorithm is the $O(nD)$ space for K and $O(n)$ for Basic,Disc.\\\\
The space complexity is $O(nD)$

\end{document}